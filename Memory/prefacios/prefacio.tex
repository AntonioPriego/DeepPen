\chapter*{}
%\thispagestyle{empty}
%\cleardoublepage

%\thispagestyle{empty}

\input{portada/portada_2}



\cleardoublepage
\thispagestyle{empty}

\begin{center}
{\large\bfseries Detección de escritura mediante Deep Learning en
un sistema empotrado\\
}
\textsc{---}\\
{\small\bfseries Deep Learning en sistemas empotrados: TinyML.}\\
\end{center}
\begin{center}
Antonio Priego Raya\\
\end{center}

%\vspace{0.7cm}
\noindent{\textbf{Palabras clave}: TinyML, Machine learning, Deep learning,
Sistemas empotrados, Reconocimiento letras, Redes neuronales convolucionales,
...}\\

\vspace{0.7cm}
\noindent{\textbf{Resumen}}\\

\begin{comment}
Creación de un dispositivo autónomo con forma de lápiz en el que integrar un
sistema empotrado, concretamente haré uso de la Arduino Nano Sense 33 BLE.
El propósito de este dispositivo será la detección de letras en tiempo
real, registrando el movimiento del dispositivo para resolver la detección
de estas.\\
Para el procesamiento y clasificación de los movimientos se empleará
deep learning, con un modelo de red neuronal convolucional. Con la
particularidad de ejecutar el procesamiento del modelo en el propio
dispositivo, para dotarlo de autonomía.\\
Por tanto se desarrollarán todos los pasos propios del trabajo con
redes neuronales: diseño del modelo, recolección de datos, entrenamiento
del modelo, etc.\\
Complementario al dispositivo, también se creará un interfaz donde acceder
a las funciones del dispositivo.
\end{comment}
\cleardoublepage


\thispagestyle{empty}


\begin{center}
{\large\bfseries Project Title: Project Subtitle}\\
\end{center}
\begin{center}
First name, Family name (student)\\
\end{center}

%\vspace{0.7cm}
\noindent{\textbf{Keywords}: Keyword1, Keyword2, Keyword3, ....}\\

\vspace{0.7cm}
\noindent{\textbf{Abstract}}\\

Write here the abstract in English.

\chapter*{}
\thispagestyle{empty}

\noindent\rule[-1ex]{\textwidth}{2pt}\\[4.5ex]

Yo, \textbf{Antonio Priego Raya}, alumno de la titulación \textit{Grado en Ingeniería Informática}
de la \textbf{Escuela Técnica Superior
de Ingenierías Informática y de Telecomunicación de la Universidad de Granada}, con DNI 31033948W, autorizo la
ubicación de la siguiente copia de mi Trabajo Fin de Grado en la biblioteca del centro para que pueda ser
consultada por las personas que lo deseen.

\vspace{6cm}

\noindent Fdo: Antonio Priego Raya

\vspace{2cm}

\begin{flushright}
Granada a DÍA de Junio de 2022
\end{flushright}


\chapter*{}
\thispagestyle{empty}

\noindent\rule[-1ex]{\textwidth}{2pt}\\[4.5ex]

D. \textbf{Jesús González Peñalver}, Catedrático del departamento de Arquitectura y Tecnología de Computadores de la Universidad de Granada.

\vspace{0.5cm}

D. \textbf{Juan José Escobar Pérez}, Profesor Sustituto Interino del departamento de Arquitectura y Tecnología de Computadores de la Universidad de Granada.

\vspace{0.5cm}

\textbf{Informan:}

\vspace{0.5cm}

Que el presente trabajo, titulado \textit{\textbf{Dispositivo para detección de escritura mediante Deep Learning en
un sistema empotrado}},
ha sido realizado bajo su supervisión por \textbf{Antonio Priego Raya}, y autorizamos la defensa de dicho trabajo ante el tribunal
que corresponda.

\vspace{0.5cm}

Y para que conste, expiden y firman el presente informe en Granada a X de mes Julio de 2022.

\vspace{1cm}

\textbf{Los directores:}

\vspace{5cm}

\noindent \textbf{Jesús González Peñalver \ \ \ \ \ \ \ \ \ \ \ Juan José Escobar Pérez}

\chapter*{Agradecimientos}
\thispagestyle{empty}

       \vspace{1cm}


Poner aquí agradecimientos...

