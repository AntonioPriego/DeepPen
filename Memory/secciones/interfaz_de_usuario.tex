\chapter{Interfaz de usuario}

En primer lugar, hice un diseño aproximado para la interfaz de usuario que
necesitaba haciendo uso de \textbf{\textit{QT Design Studio}}. Obteniendo el
siguiente resultado:

\begin{figure}[h]
    \centering
    \includegraphics[width=1\textwidth]{capturas/DisenoUsuario1.png}\\[-0,40cm]
    \caption{Primer boceto en QT Design Studio}
    \end{figure}

Con este primer acercamiento, dispuse los elementos que pensé imprescindibles.
No obstante, al finalizar esta primera iteración de diseño, pude ver que las
carencias a nivel estético; facilmente resolubles cuando comience con la
implementación real en \textbf{\textit{QT Creator}}. También vi que faltaban
algunos botones como el de sincronización con el \textbf{\textit{DeepPen}},
que \textit{Read Mode} no era un buen nombre para indicar el comportamiento
del modo, o que había un espacio en el centro de la barra superior sin usar y
que podía usar para indicar el modo actual.

\pagebreak

Teniendo estas apreciaciones en mente, las corregí en la siguiente iteración
de aproximación al diseño con el que partir en la implementación.

\begin{figure}[h]
    \centering
    \includegraphics[width=1\textwidth]{capturas/DisenoUsuario2.png}\\[-0,40cm]
    \caption{Segundo boceto en QT Design Studio}
    \end{figure}

Para pasar esta aproximación de diseño a una implementación real, utilicé
\textbf{\textit{QT Creator}} con \textbf{\textit{C++}}. Resultando en lo siguiente:

\begin{figure}[h]
    \centering
    \includegraphics[width=1\textwidth]{capturas/Interfaz.png}\\[-0,40cm]
    \caption{Primera implementación de la interfaz de usuario del
    \textbf{\textit{DeepPen}}}
    \end{figure}

