\chapter{Diseño del sistema}
Para poder trabajar con objetivos claros, lo mejor es definir la estructura
de nuestro proyecto, constreñir los elementos que constituirán el producto
que se trata de alcanzar.

\section{Estructuración del diseño del sistema}
Lo primero es identificar los elementos. Es evidente que nuestro
dispositivo estará integrado en un encapsulado, por lo tanto, cuando se haga
referencia al \textit{SmartPen}, se estará haciendo alusión al propio encapsulado
que contiene toda la electrónica agregada. Dicha electrónica consta de dos partes
físicamente separadas: el microcontrolador y la batería que lo alimentará
cuando se haga uso de su característica inalámbrica (bluetooth), dotándolo de
la autonomía necesaria. Nuestro \textit{SmartPen} necesitará de un equipo en el
que mostrar las funcionalidades que ofrece el producto, podría tratarse de, por
ejemplo un smartphone, pero en este caso se ha optado por hacerlo en un ordenador
para agilizar el desarrollo. El ordenador es una herramienta que será utilizada
no solo cuando se concluya el desarrollo, como interfaz para el \textit{SmartPen},
sino como herramienta para la propia producción de todo el software requerido
durante el desarrollo del proyecto:
interfaz de usuario, firmware del micrococontrolador y creación de la red
neuronal. A su vez, el firmware hará uso de varios elementos clave para su
funcionamiento; ya que la red neuronal tomará como entrada imágenes, se
necesitará de una parte del firmware destinada a rasterizar el movimiento,
movimiento que por otro lado deberá registrarse por medio de los sensores
presentes en el microcontrolador. Finalmente, el dispositivo requerirá
según lo especificado (sección \ref{reqNF}), de mínimo dos canales de
comunicación uno inalámbrico y otro por cable. Canales de comunicación
que no solo se emplearán con fines de conexión con la interfaz de usuario,
sino que también servirán para generar las muestras con las que se
desarrollará (validación, entrenamiento y testeo) la red neuronal.

\begin{figure}[]
    \centering
    \includegraphics[angle=90,width=1\textwidth]{capturas/EstructuraDiseño.png}\\[-0,20cm]
    \caption{Esquema de la estructura del diseño del sistema}
\end{figure}
