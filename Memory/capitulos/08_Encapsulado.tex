\chapter{Encapsulado\label{capEncapsulado}}
Para esta parte, dada la menor relevancia al poner en contexto todo el proyecto, se
contará con poco tiempo. Esto supondrá decisiones tomadas con esta limitación,
\section{Herramientas utilizadas}
Para la creación de los modelos 3D del encapsulado, se han utilizado principalmente
dos herramientas distintas para modelar y una para imprimir.

Para el modelado se utilizará \textit{Blender}, un software de modelado muy potente
y lleno de posibilidades, aunque considerablemente complejo al empezar y que se utilizará
para no limitar el modelo a algo simple. Fuera de lo
planificado, también se ha utilizado \textit{SketchUp}, por los motivos que se
desarrollan en Problemas \ref{problemaBateria}.2.

Para la impresión, de la que yo no me encargaré, se hará uso del \textit{slicer}
(software para impresión 3D) \textit{Ultimaker Cura}.

\section{Implementación}
Por la limitación de tiempo descrita al comienzo de este Capítulo \ref{capEncapsulado},
no se dispone de tiempo para una fase de diseño previa a implementar el modelo.
El modelo se ha construirá con las especificaciones más básicas en mente.

El encapsulado debe contener de forma firme el microcontrolador para que no se mueva,
ya que se espera que el usario lo mueva con determinación para trazar una letra. Por
otro lado, también será necesario incorporar una batería para alimentar el dispositivo
sin conexión directa al ordenador. Con estas dos únicas limitaciones y con una forma
semejante a la de un bolígrafo, se construirá el modelo.

El modelo estará dividido en componentes por motivos de la logística de la impresión, tanto
por comodidad para tratar con la persona que lo imprimirá, como por cuestiones de que los
modelos deben guardar ciertas limitaciones estructurales para poder imprimirse. Por ejemplo,
en lugar de imprimir el
\href{https://github.com/AntonioPriego/SmartPen/tree/main/SmartPenModel/Components/cilindroBajo}{cilindroBajo}
ya con la 
\href{https://github.com/AntonioPriego/SmartPen/tree/main/SmartPenModel/Components/punta}{punta},
para aunar toda la parte inferior del encapsulado, se imprimen por separado, ya que se necesita de
una base estable para la impresión que no se lograría con los dos elementos unidos.

\begin{problemas}{Modelo incompatible con la impresión}
    \color{mitexto}
    Durante la impresión surgieron varios problemas, la mayoría menores.
    Sin embargo hubo uno que se resistió, aunque no entrañaba realmente ninguna complejidad.
    Este era que una de las partes del encapsulado, el
    \href{https://github.com/AntonioPriego/SmartPen/tree/main/SmartPenModel/Components/cilindroBajo}{cilindroBajo},
    la parte que contendrá al microcontrolador, mostraba en el \textit{slicer} (software
    para la impresión), un error de formato en la base. Este error finalmente se debía
    a un grosor por debajo del mínimo. Problema ocasionado por haber construido
    este componente a partir del escalado vertical de otro
    (\href{https://github.com/AntonioPriego/SmartPen/tree/main/SmartPenModel/Components/cilindroAlto}{cilindroAlto});
    al escalar verticalmente, el grosor de la pared del modelo, se conserva, sin embargo
    no ocurre lo mismo para la base, disminuyendo su grosor y provocando este problema.
    La solución fue simplemente redefinir el grosor de la base.
\end{problemas}

Con todos los componentes impresos, surge un problema con la batería con la que contaba
para incorporar en el encapsulado

\begin{problemas}{Alimentación intermitente de la batería\label{problemaBateria}}
    \color{mitexto}
    La alimentación resultaba deficiente, por lo tanto solo pude hacerme con
    otra batería y adaptar el encapsulado a este nuevo componente.
    Para poder crear el modelo complementario para adaptar la batería, hice uso
    de \textit{SketchUp}, un software muchísimo más simple que \textit{Blender};
    esta sencillez ha sido determinante porque el tiempo era muy restrictivo en ese momento.
    Resultando en el componente
    \href{https://github.com/AntonioPriego/SmartPen/tree/main/SmartPenModel/Components/slotBateria}{slotBatería},
    para poder incluir una pila de 9V recargable. La alimentación con esta pila, al ser de más
    de 5V, deberá ser vía pines de la placa, que admiten hasta 21V.
\end{problemas}

Con esta nueva batería, se logra una autonomía de más de dos días en el primer ciclo
de carga. Según el fabricante, los primeros ciclos de carga suelen ser los que menor
energía suministran, por lo que podría incluso mejorar la duración.

La batería se conecta directamente a los pines de alimentacion (\textit{VIN})
y masa (\textit{GND}) del microcontrolador.

En los Apéndices \ref{modelos} y \ref{encapsulado} pueden encontrarse ilustrados,
respectivamente, los modelos creados y el resultado final del encapsulado con
todo el hardware integrado.