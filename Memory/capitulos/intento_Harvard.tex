%\begin{mimargen}{-1cm}{-1cm}

\chapter{Segundo acercamiento}
Comienzo ejecutando tal cual el código del repo
\url{https://github.com/petewarden/magic_wand}.

Importante tener la biblioteca de tensorflow actualizada(en este
caso la provista por pete warden).

Visto cómo toma pete los datos, pruebo a hacerlo.
Cargar el programa de recolección de datos en la placa(creo que funciona el magic\_wand,
PROBAR).
Acceder a la web de recolección. Recolectar.
Para acceder a la web, hay que hacerlo con chrome y con la flag
'Experimental Web Platform features' activa en 'chrome://flags/'


Hay un error con el DataCollector, además de que asigna mal los índices(index)
al borrar muestras, no guarda todas las muestras que tomas, da la sensación de que
hay un límite.
Para contar el número de muestras que hay en el JSON, uso la siguiente línea bash:
\begin{lstlisting}[language=bash]
~$ grep -o index <nombre_archivo>.json | wc -l
\end{lstlisting}



\textbf{CAMBIAR ORIENTACIÓN DE LA PLACA} \\
Como no hay referencias de la orientación original de los sensores en ningún
datasheet, con el propio sketch que hice originalmente para tomar las muestras,
comprobé la posición con la que funciona, la comparé con la posición que quiero
tener y gracias a eso pude hacer el siguiente cambio.\\
Lo mejor que he conseguido cambiando en la biblioteca de los sensores,
en LSM9DS1.cpp en todas los sensores:
\begin{lstlisting}[language=c++]
//          Original         //     Orientacion cambiada
x = data[0] * 4.0 / 32768.0; // z = -data[0] * 4.0 / 32768.0;
y = data[1] * 4.0 / 32768.0; // y =  data[1] * 4.0 / 32768.0;
z = data[2] * 4.0 / 32768.0; // x =  data[2] * 4.0 / 32768.0;
\end{lstlisting}

\newpage
Creo que habría que cambiar algo de deep\_pen.ino(a partir de línea 408)
\begin{lstlisting}[language=c++]
    const float gy = current_gravity[1];
    const float gz = current_gravity[2];
    float gmag = sqrtf((gy * gy) + (gz * gz));
    if (gmag < 0.0001f) {
      gmag = 0.0001f;
    }
    // ...
\end{lstlisting}


\textbf{LA OBTENCIÓN DE DATOS TIENE UN FALLO AL BORRAR MUESTRAS} \\
Este fallo aparentemente se produce al borrar instancias por debajo de
la última. En ocasiones, se produce un pequeño error en las comprobaciones
de índices de muestras, por lo que se borran varias muestras y estas terminan
con índices desordenados.\\
Para paliar este problema, he creado un pequeño script que toma el número de
muestras y ordena los índices de las mismas.\\

\textbf{MODELO}\\
Se trata de un modelo de aprendizaje supervisado, es decir, que está basado en
etiquetas(\textit{labels}) estas etiquetas representarán las distintas soluciones
a las que hará frente el modelo; en nuestro caso, letras.
La forma que emplearemos para representar dichas letras como input para el modelo,
serán imágenes. Aunque para tomar dichas imágenes, hacemos uso del giroscopio
y el acelerómetro de la placa.\\


\textbf{LEER DEL PUERTO SERIE EN QT}
Hay que añadir al *.pro del proyecto QT:
\begin{lstlisting}[language=make]
  QT += serialport
\end{lstlisting}
Y tras esto, añadir la biblioteca con normalidad y acceder al puerto
con el nombre, en nuestro caso, "ttyACM0".\\\\

\textbf{FALLO RECONOCIMIENTO DE IMÁGENES QT}\\
Al importar imágenes en QT tras haber exportado el programa a Win y Linux para
probar que todo fuera correctamente, las imágenes dejaron de mostrarse.\\
Esto se debe a que cambió la ruta del proyecto al directorio en el que se
exportó. Por tanto las rutas especificadas para las imágenes, dejaron de ser
válidas. Para corregir este problema, basta con cambiar el 'Build directory' del
proyecto(Desde 'Projects' en el panel de la derecha del QT creator).\\

\textbf{HEBRAS QT}\\
Para la lectura del puerto serie desde el programa, necesitamos imporat

\textbf{VISOR WEB QT}\\
Para hacer empotrar html en qt, importamos las bibliotecas de QWeb. Para ello,
necesitamos aantes instlar todo el QtWebKit siguiendo las instrucciones:
\url{https://github.com/OpenBoard-org/OpenBoard/wiki/Build-OpenBoard-on-Ubuntu-20.04}

\textbf{VINCULAR PUERTO SERIE A PLACA}\\
Para ello, haré uso de las reglas de udev del sistema linux.
Lo primero es conocer la información de nuestra placa:
\begin{lstlisting}[language=make]
  ~$ lsusb
[...]
Bus 003 Device 004: ID 2341:805a Arduino SA Nano 33 BLE
[...]
\end{lstlisting}
De aquí podemos extraer el fabricante(o idVendor) y el producto de este fabricante
(o idProduct). Necesitaremos ambos.

Ahora necesitamos el serial del puerto al que vincularemos la placa:
\begin{lstlisting}[language=make]
  ~$ udevadm info -a -n /dev/ttyACM0 | grep serial
ATTRS{serial}=="185F25FD3EF48040"
\end{lstlisting}
Con esto, ya podemos crear la regla para vincular univocamente el puerto a la placa
y evitar así problemas con la detección en el UI.\\
En /etc/udev/rules.d/99-ftdi.rules (en mi caso, varía dependiendo del sistema),
crearemos la regla \\SUBSYSTEM=="tty", ATTRS{idVendor}=="2341", ATTRS{idProduct}=="805a",
ATTRS{serial}=="185F25FD3EF48040", SYMLINK+="ttySLAB0"\\por los datos obtenidos.\\

\textbf{CAMBIOS MODELO}\\
Cambiar el tamaño del kernel de las capas Conv2D de 3 a 4, resulta muy efectivo,
a costa, evidentemente, de aumentar el tamaño que ocupa el modelo.\\
Al pasar de 4 a 5, el modelo arroja unos datos de efectividad teóricos extremadamente
buenos, alcanzando cifras de precisión mucho más altas con menos epochs.
En general, podemos extrapolar que, a mayor tamaño del kernel, mejor precisión pero
mayor tamaño del modelo. En nuestro caso no llega a ser un problema, ya que, aunque
estamos usando un dispositivo de memoria limitada, no llega a ocuparse toda la memoria
del mismo, al menos por ahora.\\

\textbf{PRUEBAS BLUETOOTH}\\
He tenido que instalar qtconnectivity5-dev para probar el QT project que estoy probando.\\

\textbf{POR QUÉ ESTA PLACA?}\\
Aduino Nano Sense 33 BLE\\
Por qué arduino: Documentación, respaldo comunidad, IDE facilita trabajo, etc.\\
Por qué Nano: Queremos integrarla en un 'lápiz', debe ser un dispositivo pequeño.\\
Por qué Sense: Necesitamos los sensores para el reconocimiento de movimiento.\\
BLE: Por pura utilidad, es mucho más cómodo utilizar el lápiz de forma inalámbrica.
Además no tiene mucho sentido integrar el procesamiento en un dispositivo pequeño
si va a depender de un ordenador.\\

\textbf{DESCONEXIÓN DEL PROGRAMA A BT AL ESCRIBIR PLACA EN RX}\\
Para el control de flujo de la comunicación bluetooth(asegurarse de que recibimos
bien y solo una vez cada letra), implemento un sistema de señales.
Cuando recibimos una letra desde la placa(mediante canal tx), el programa la almacena y
envía una señal de que ha recibido la letra(mediante canal rx) y cuando la placa
recibe esta señal, borra del canal tx la letra para que no vuelva a leerse desde el
programa.
El problema descrito, viene, creo, al escribir en rx la señal de recibo. Por algún
motivo, el programa se desconecta de la placa.\\Finalmente no se debe a esto.\\


\textbf{PROBLEMAS UI-BLUETOOTH}\\
Desconexiones random cada cierto tiempo durante lectura BLE.
Lecturas dobles al escribir durante reconexión.

\textbf{BUFFER PARA SINCRONIZACIÓN DE ESCRITURA AUTÓNOMA}\\

\textbf{MÉTODO DE ENVÍO DE BUFFER DE LETRAS PARA CONEXIÓN BT CUANDO HAYA UNA CADENA
PREVIA A CONEXIÓN}\\

~\\~\\~\\Documentación Bluetooth LE:
https://doc.qt.io/qt-6/qtbluetooth-le-overview.html

~\\~\\~\\Visualizar red neuronal :
http://alexlenail.me/NN-SVG/index.html

~\\~\\~\\Documentación sobre Keras:
https://keras.io/api/

%\end{mimargen}