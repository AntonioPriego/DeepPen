\chapter{Introducción y Motivación}
Como consecuencia del desarrollo de la informática, y la expansión y
filtración del uso de equipos informáticos en la población general,
cada vez la escritura tradicional cae más en desuso. Y es que una
vez se supera la etapa académica, pocas personas siguen utilizando
en su cotidianidad la escritura manual. Incluso para la educación
hay un creciente movimiento de adaptación tecnológica que releva
cada vez más al lápiz y papel.\newline
No es el objetivo pecar de romanticismo y mirar a través de la lente
de la nostalgia, sino avanzar, eso sí, intentando conservar en el proceso
de innovación, las técnicas que nos han traído hasta aquí.

Por lo que el motivo de este trabajo siempre ha sido tratar de,
creando nueva tecnología, favorecer el uso de la escritura.
Ya que el avance y el progreso es no solo imparable sino necesario,
la única forma de preservación de la grafía manual es establecer
alternativas modernas que complementen a los dispositivos ya
constituidos y que empleamos en el día a día.

Gracias al avance tecnológico de décadas, hoy podemos contar con
herramientas informáticas de gran capacidad como lo son todos
los mecanismos de Inteligencia Artificial. Concretamente el Deep
Learning y las redes neuronales son conceptos en gran expansión
durante los últimos años.
El \textit{Deep Learning} es un campo que está cambiando
la informática como la concebíamos, revelándose como una alternativa
sobresaliente para problemas que trabajan con grandes volúmenes de
información, que presentan una elevada complejidad o simplemente
que cumplen mejor de lo que habituaban a hacerlo con técnicas
previas. Respondiendo oportunamente al contexto temporal vigente
donde el \textit{Big Data}, \textit{Data Science}, automatización
de tareas cotidianas, detonación de herramientas y dispositivos
inteligentes, humanización de robots y robotización de personas;
perfilan y caracterizan la fase en la que nos encontramos.
Hecho que nos lleva al interés por un terreno tan intrincado
como útil y repleto de potencial. Un potencial evidenciado por
las tantas aplicaciones con sobrecogedores resultados que hace pocos
años casaban más con la ciencia ficción que con algo alcanzable, y que
emplean esta herramienta y que serán citadas a lo largo de este trabajo.


Por sus demostradas altas capacidades para
la clasificación en el procesamiento de imagen, por el hito
que supuso integrar redes neuronales en sistemas tan reducidos,
porque hay algo sugestivo en el hecho de rescatar lo tradicional
mediante las técnicas más incipientes,
pero por encima de todo, por lo estimulante que resulta trabajar con
estos mecanismos y que es algo que siempre ha rondado entre mis pensamientos;
este trabajo consistirá
en trasladar a la realidad una alternativa moderna a la escritura
manual, haciendo uso de Deep Learning en un sistema empotrado
para mantener autonomía.

Los usos pueden ser los que se deseen y se alcancen a imaginar;
con pocos añadidos podría convertirse en una herramienta para
introducir a personas de avanzada edad al manejo de ordenadores, reduciendo
la barrera de entrada al tener una forma de interactuar que ya les es familiar;
en un instrumento para hacer más ameno y dinámico el proceso de aprender
a escribir para niños y niñas; en un cuaderno virtual en el que anotar cuanto queramos sin necesidad de
transportar el medio en el que se escribe; incorporando una punta
con grafito o tinta, podríamos transcribir digitalmente lo que escribimos
en cada momento de manera física, es decir, una copia digital, un registro
de lo que hemos escrito; etc.

\begin{comment}
Pero por qué no unir ambas experiencias para obtener la sensación de
escritura tradicional y la pretensión pragmática de utilizar nuevas
tecnologías.\\

Es lo que se plantea en este proyecto, un dispositivo que, haciendo
uso de las técnicas más modernas de inteligencia artificial, ayude a
la preservación de la escritura manual; facilitando la inclusión de
la misma en el uso ordinario de los dispositivos electrónicos.\\

Asimismo, en algunos casos sería una forma de incentivar e introducir
a personas poco habituadas al uso de la tecnología, ya que el hecho
de enfrentarse a un nuevo instrumento, puede suponer una barrera
psicológica al empezar para personas de avanzada edad.\newline
Otra posibilidad es enfocarlo como herramienta didáctica, para
que los niños aprendan a escribir de una forma divertida y atractiva.\\ 

Por tanto, el objetivo de este proyecto es desarrollar no solo
un dispositivo autónomo de captación de escritura manual, sino
un entorno completo con la finalidad de diluir la barrera entre
lo tradicional y lo moderno.
\end{comment}