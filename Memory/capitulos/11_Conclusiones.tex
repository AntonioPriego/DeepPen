\chapter{Conclusiones}
Las conclusiones tras el desarrollo del proyecto y ver los resultados, son que
es algo de lo que estaría orgulloso si lo hubiese visto al comienzo del mismo.
Sin embargo no puedo evitar sentir que, por todo lo que tenía en mente
haber implementado, podría haber sido incluso mejor. Aunque ya pude suponer
desde el principio que el tiempo del que disponía no era suficiente para llenar
de funcionalidades el proyecto, por eso planteé lo que queda descrito en esta
memoria de la forma que lo hice y he cumplido con ello de forma satisfactoria.

Por otro lado, el haber trabajado con \textit{Deep Learning}, \textit{redes neuronales},
\textit{capas}, \textit{entrenamientos} y otros tantos conceptos que me resultaban tan llamativos
e interesantes, y haberlo podido unir al campo del \textit{hardware} que tanto
me apasiona, mediante la integración del procesamiento en un microcontrolador,
es otra de las razones por las que me complace haberme decidido por este proyecto.

Ha sido un trabajo complicado desde su propio planteamiento, los complejos mecanismos
que se emplean, el no haberlos utilizado nunca antes y por los problemas
que han ido surgiendo durante el desarrollo, fruto de la implementación.
Pero el resultado y el aprendizaje surgido del esfuerzo para poder completarlo,
han hecho que valga la pena.

Finalmente, es necesario concluir que este no es el final de este proyecto,
planeo continuar con su desarrollo como pasatiempo y me llenaría de orgullo
que otras personas interesadas en estos campos, participaran también
agregando sus aportaciones como se plantea al hacer todo el proyecto
público y abierto.