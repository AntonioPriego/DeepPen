\begin{mimargen}{-0.65cm}{-1cm}
\chapter{Entrenamiento}

Para comenzar con Arduino+TensorFlow, instalaré la bibliotecas
necesarias para usar TensorFlow y la \textit{Nano 33 Sense}:
\begin{enumerate}
    \item \textbf{Arduino\_TensorFlowLite}: Permite construir aplicaciones con
    aplicaciones para AI/ML.
    \item \textbf{Arduino\_LSM9DS1}: Provee de las herramientas para acceder al
    acelerómetro, magnetometro y giroscopop del \textit{Nano 33 BLE Sense}.
\end{enumerate}

\cite{github-magicwand}Además, hay que realizar una serie de cambios en la biblioteca
\small\textbf{LSM9DS1.cpp}.\normalsize\\
Añadir justo antes de que la función \small\textbf{LSM9DS1Class::begin()}\normalsize
~retorne:
\begin{lstlisting}
    // Enable FIFO (see docs https://www.st.com/resource/en/datasheet/DM00103319.pdf)
    writeRegister(LSM9DS1_ADDRESS, 0x23, 0x02);
    // Set continuous mode
    writeRegister(LSM9DS1_ADDRESS, 0x2E, 0xC0);
\end{lstlisting}

También debemos editar \small\textbf{LSM9DS1Class::accelerationAvailable()}
\normalsize:
\begin{lstlisting}
    int LSM9DS1Class::accelerationAvailable()
    {
        /********************************** OLD *********************************
        if (readRegister(LSM9DS1_ADDRESS, LSM9DS1_STATUS_REG) & 0x01) { 
            return 1; 
        }
        *********************************** OLD *********************************/
    
        // Read FIFO_SRC. If any of the rightmost 8 bits have a value, there is data
        if (readRegister(LSM9DS1_ADDRESS, 0x2F) & 63) {
            return 1;
        }
        
        return 0;
    }
\end{lstlisting}

Si queremos acceder al puerto serie de la placa o usar el IDE de Arduino, debemos
conceder permisos al dispositivo:
\begin{lstlisting}[language=bash]
    ~$ ls -l /dev/ttyACM*                     # En mi caso
    ~$ sudo usermod -a -G dialout <usuario>
\end{lstlisting}~\\


Cuando ya tenemos la biblioteca para los sensores y el acceso al puerto garantizado,
podemos pasar a probar el programa.
Si funciona con el entrenamiento por defecto, continuamos finalmente con nuestro
entrenamiento.\\
Realizaremos el entrenamiento recogiendo muestras para cada caracter a
reconocer. Una vez tenemos suficientes muestras para todos los caracteres, podemos
realizar el entrenamiento del modelo, que realizaremos en \textit{Google Colab}.
El script de entrenamiento es el siguiente: [\textbf{METER ENLACE}].

~\\
Toma de muestras, para el software que utilizaremos para tomar los datos
para crear el dataset con el que entrenar el modelo, necesitaremos instalar las
siguientes bibliotecas:\\
\cite{apds9960} Arduino\_APDS9960: para disponer de librerías para algunos sensores adicionales.\\
\cite{cmsisdsp} Arduino\_CMSIS\-DSP: para disponer de arm\_math.h.\\
\cite{lps22hb} Arduino\_LPS22HB: herramientas para disponer del sensor de presión.\\
\cite{hts221} Arduino\_HTS221: herramientas para el sensor de temperatura y humedad.

\end{mimargen}