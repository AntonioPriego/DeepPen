\chapter{Trabajos futuros y mejoras}
Si bien el dispositivo realiza lo que se postulaba al comienzo
como requerimientos. Sin embargo quedan muchas mejoras y ampliaciones
posibles, en su mayoría por carencia de tiempo.

Una de las que más me incordian, es el no haber podido añadir
la funcionalidad de 'modo pizarra', ya que si bien no estaba planteado
como requisito, era algo que pretendía hacer desde el planteamiento del
trabajo, ya que me resulta especialmente útil como complemento a una
herramienta que pretende ser un híbrido entre lápiz y teclado.
De hecho la solución era integrar el módulo de exposición del trazado
de la herramienta de recolección de datos
(\href{https://github.com/AntonioPriego/SmartPen/blob/main/DataCollector/SmartPen_DataCollector.html}{\textit{SmartPen\_DataCollector}})
en el propio interfaz de usuario, haciendo uso de las herramientas de \textit{QT}
para visualización web. Sin embargo al emplear la web \textit{bluetooth}
para la recolección del trazo generado por el microcontrolador, dificultaba
mucho la implementación y viendo que se alargaría más de lo esperado,
se tomó la decisión de postergarlo.

Otro añadido que habría dado más autonomía al dispositivo, es el uso
de un buffer de memoria en el microcontrolador para almacenar
las letras que se escribieran previas a la conexión con la interfaz
de usuario. Y que es en realidad una funcionalidad muy fácil de implementar
ya que solo habría que crear el propio buffer en el microcontrolador
y cambiar la lógica de la comunicación interfaz-microcontrolador de una
letra a varias letras consecutivamente enviadas.

Esta sin embargo sería una mejora mucho más compleja y que llevaría
un proceso de documentación a bajo nivel, un tanto distendido.
El problema de desconexiones descrito en Problemas \ref{errDescBT}.
La solución sería encontrar el problema que desemboca en la desconexión
y que parte de la librería mencionada y el stack de protocolos \textit{bluetooth}
de \textit{linux}.

Ocasionalmente la interfaz de usuario sufre problemas de corrupción de memoria.
Son tan raros que trazarlos es muy complicado y no he sido capaz de entender la
causa. Este es un problema que queda pendiente a resolver al no disponer de más
tiempo.
Para resolverlo, el primer paso podría ser utilizar \textit{Valgrind}
para dar con el origen del problema, que con total seguridad, será de nuevo,
algo respectivo a la librería mencionada o a la gestión de \textit{QThread}.

Lo cual nos lleva a otra posible mejora y es la implementación de la interfaz
para \textit{Windows} o \textit{macOS}. Me gustaría poder haberla realizado, pero el tiempo
no alcanza para más y dado que trabajo habitualmente en \textit{linux},
era más inteligente comenzar implementando la interfaz solo para este sistema.
En principio la solución pasaría por, simplemente, añadir la configuración de
puertos correspondientes para cada sistema.

Una mejora que significaría un rango mucho mayor de detección para el \textit{SmartPen}
y algo más de precisión, sería aumentar el número de muestras con el que
entrenar el modelo. Pero como se expuso en la Sección \ref{dataColl}, es un proceso
que lleva muchísimo tiempo. Aunque sencilla, esta mejora supone emplear tiempo
del que no se dispone.

En realidad hay un sinfín de posibles mejoras y nuevas funcionalidades,
pero por esto mismo y dado que es un proyecto interesante y lleno de
posibilidades, se ha planteado su desarrollo abierto y libre a aportaciones de
la comunidad.