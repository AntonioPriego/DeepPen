\chapter{Validación}

\section{Ajuste al presupuesto}
Respecto a como se estimó en la sección \ref{presupuesto} (\textit{Presupuesto}),
se ha ajustado satisfactoriamente el coste del dispositivo, resultando todos los costes
los reflejados en la Tabla \ref{tabPresFin}.
\begin{table}[h]
    \begin{tabular}{ll}
    \hline
    \rowcolor[HTML]{6665CD} 
    \multicolumn{1}{|l|}{\cellcolor[HTML]{6665CD}{\color[HTML]{EFEFEF} \textbf{Descripción}}} & \multicolumn{1}{l|}{\cellcolor[HTML]{6665CD}{\color[HTML]{EFEFEF} \textbf{Precio}}} \\ \hline
    \multicolumn{1}{|l|}{Arduino Nano Sense 33 BLE ~~~~~~~~~~~~~~~~~~~~~~~~~~~~~~~~~~~~~}& \multicolumn{1}{r|}{35'80\$$\sim$33'82€}                                            \\
    \multicolumn{1}{|l|}{Pila 9V Recargable}                                                  & \multicolumn{1}{r|}{10'99€}                                                         \\
    \multicolumn{1}{|l|}{Adaptador pila 9V}                                                           & \multicolumn{1}{r|}{3€}                                                             \\
    \multicolumn{1}{|l|}{Impresión}                                                           & \multicolumn{1}{r|}{1€}                                                             \\
    \multicolumn{1}{|l|}{Interruptor}                                                           & \multicolumn{1}{r|}{0'05€}                                                             \\
    \multicolumn{1}{|l|}{Adaptador microUSB a USB}                                                           & \multicolumn{1}{r|}{1€}                                                             \\
    \multicolumn{1}{|l|}{Cable MicroUSB Datos}                                                & \multicolumn{1}{r|}{6€}                                                             \\ \hline
    \multicolumn{1}{r}{\textbf{TOTAL:}}                                                       & \textbf{55,86€}                                                                    
    \end{tabular}
    \caption{Costes de producción del \textit{SmartPen}\label{tabPresFin}}
\end{table}

Respecto a las horas de trabajo, también se ha respetado la estimación de
la planificación, ya que he tratado de ceñirme a estas durante todo el desarrollo del
proyecto.

\section{Comprobación de objetivos cumplidos}
Los requisitos para el proyecto y sus derivadas especificaciones, han sido
satisfactoriamente cumplidas.

El presupuesto se ha ajustado a lo estimado tanto en su aspecto de trabajo
como en el de ajuste para los fondos empleados en el \textit{hardware}.

El microcontrolador efectivamente, como se propuso, hace uso de un modelo
basado en \textit{Deep Learning} y además lo hace de manera eficaz. Para
alimentar al modelo de procesamiento se hace uso de los sensores que ofrece
el microcontrolador, el cual cuenta con dimensiones muy reducidas para
poder acoplarse en un encapsulado embellecedor que dará cabida también
a una batería que lo alimentará para, como se especifica, dotarlo de autonomía.
Autonomía necesaria ya que el dispositivo funciona tanto por cable
como de forma inalámbrica, tal y como se demandaba.

Todo el desarrollo es transparente y de código abierto, por lo que
está acondicionado para que otros desarrolladores y usuarios hagan uso del progreso logrado
en este despliegue del producto y aporten funcionalidades al producto
si así lo consideran.

Todo el \textit{software} utilizado cumple con las especificaciones, porque de otra forma,
no podría haberse llevado a cabo el proyecto.
La interfaz de usuario cumple con la funcionalidad mínima propuesta;
el \textit{framework} para el diseño, entrenamiento, validación y testeo
de la \textit{red neuronal} es el mejor que podría haberse empleado;
el \textit{firmware} del microcontrolador ejecuta exactamente lo que
se ideó al comienzo del proyecto; y respecto al \textit{software} para 
el modelado del encapsulado, se han trabajado con varias alternativas
en función de lo que se requería en cada momento.

En general el producto no solo realiza su cometido, sino que lo hace
de forma satisfactoria.
{\color{red} Añadir 'como muestra tal' si da tiempo a demo}